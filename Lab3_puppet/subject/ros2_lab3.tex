\documentclass{ecnreport}

\stud{Master 1 CORO / Option Robotique}
\topic{Robot Operating System}
\author{O. Kermorgant}

\begin{document}

\inserttitle{Robot Operating System}

\insertsubtitle{Lab 3: the ``puppet arm'' node}

\section{Goals}

In this lab, again the left arm will be controlled depending on the position of the right arm.\\
This time, a constant 3D transform will be imposed between the left and right grippers. 

\subsection{Bring up Baxter}

Even if you have a real Baxter robot it can be a good idea to test the lab in simulation first.
In both cases, we want to have a RViz display, which is mandatory in simulation and quite handy on the real robot. RViz is run automatically in both cases.

\subsubsection{On the real robot}

Baxter is a ROS1-based robot. To work with ROS 2 we thus have to run a bridge that transforms all or some topics between ROS 1 and ROS 2.

A launch file is available in the \okttt{baxter_bridge} package to run both the bridge and RViz.

You have to connect to Baxter's ROSMASTER in the terminal where you run the bridge:
\begin{bashcodelarge}
 ros2ws && ros_baxter # so that your ROSMASTER is Baxter
 ros2 launch baxter_bridge baxter_bridge_launch.py
\end{bashcodelarge}

\subsubsection{In simulation (including virtual machine users)}

The Baxter simulator behaves as the actual Baxter from the ROS 2 side, only with a very limited part of the same topics and services. 

The \okttt{baxter_bridge} node should be run from a ROS 2 terminal:
\begin{bashcodelarge}
ros2ws
ros2 launch baxter_simple_sim sim_launch.py lab:=puppet
\end{bashcodelarge}
The last argument makes the right arm move with a pre-computer motion, as we cannot move it manually.

\subsection{Initial state}

The right arm of the robot can be manually moved by grabbing the wrist, or has a pre-computed motion (in simulation). 
It is waiting for any command for the left arm.\\

On the real robot of course, only one group can run their code at a time. This is ensured by the \okttt{baxter_bridge} that only republished command topics if no other computer has just published on the same topic. Thus, it may be interesting to test your code in the simulation first.

A few in/out topics exist and can be listed through:
\begin{bashcodelarge}
rostopic list (ROS 1)
ros2 topic list (ROS 2)
\end{bashcodelarge}


\section{ROS concepts}

\subsection{Publishing to a topic}

In order to control the left arm, you need to publish a command message on the suitable topic (as in the lab 2)

\subsection{Using TF}

In ROS, the \okttt{/tf} topic conveys many 3D transforms between frames, forming a tree. A TF listener can be instantiated in a node in order to retrieve any transform between two frames. \\

For \okttt{/tf} to get all the transforms for Baxter, a \okttt{robot_state_publisher} must be run. It is a standard node that:
\begin{itemize}
 \item loads the description of a robot (URDF file)
 \item subscribes to the joint states topic
 \item publishes all induced 3D transforms with the direct geometric model
\end{itemize}
In this lab, Baxter's \okttt{robot_state_publisher} is embedded either in the simulator or in the bridge and automatically subscribes to the joint states.

\subsection{Adding a new frame}

The desired pose of the left gripper will be defined as a new frame, relative to the right gripper and called \okttt{'left_gripper_desired'}.\\
The node \okttt{static_transform_publisher} from the \okttt{tf2_ros} package is designed to publish this kind of arbitrary fixed transform between frames:
\begin{bashcodelarge}
 ros2 run tf2_ros static_transform_publisher x y z yaw pitch roll frame_id child_frame_id
\end{bashcodelarge}where:
\begin{itemize}
\item \okttt{frame_id} is the reference frame (here \okttt{right_gripper})
\item \okttt{child_frame_id} is the target frame (here \okttt{left_gripper_desired})
 \item \okttt{x y z yaw pitch roll} is the 3D transform (here \okttt{0 0 0.1 0 3.14 0})
\end{itemize}

\subsection{Putting all together}

A single launch file should be written in order to regroup the previous commands:
\begin{itemize}
 \item run the node \okttt{static_transform_publisher} with the correct arguments
 \item include the launch file \okttt{robot_state_publisher_launch.py} from the \okttt{baxter_description} package
 \end{itemize}
 
 With the simulation and your launch file running, check that you can indeed retrieve the current 3D transform between the frames \okttt{base} and \okttt{left_gripper_desired}:
 \begin{bashcodelarge}
 ros2 run tf2_ros tf2_echo base left_gripper_desired
\end{bashcodelarge}

\section{Tasks}

\begin{itemize}
\item Identify the topics that the node should publish to: names, message type
\item Identify the services that the node need in order to convert 3D transform to joint positions: names, service type
\item Check the online documentation ``ROS 2 C++ services'' to get the overall syntax. This work requires at least a publisher, a timer, tf, and service client
\item Program the node in C++ (and then in Python3 if you feel like it)
\end{itemize}

The package is already created for this lab (\okttt{lab3_puppet}), you just have to update the C++ file and compile it.\\
Feel free to keep this package as a template for future packages / nodes that you will create.
\end{document}
